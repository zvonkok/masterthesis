%*******************************************************
% Abstract
%*******************************************************
\chapter*{Abstract}
\renewcommand{\chapterpagestyle}{empty}
% Motivation: Why do we care about the problem and the results? If the 
% isn't obviously "interesting" it might be better to put motivation first; 
% but if your work is incremental progress on a problem that is widely 
% recognized as important, then it is probably better to put the problem 
% statement first to indicate which piece of the larger problem you are breaking
% off to work on. This section should include the importance of your work, the 
% difficulty of the area, and the impact it might have if successful.#
Today's \glspl{GPU} are not only good for gaming and graphics processing their
highly parallel structure is predestined for a range of complex algorithms. They
offer a tremendous memory bandwidth and computational power. Contrary to
\glspl{CPU}, \glspl{GPU} are accelerating quickly and advancing at incredible
rates in terms of absolute transistor count. Implementing a massively parallel,
unified shader design, its flexibility and programmability makes the \gls{GPU}
an attractive platform for general purpose computation. Recent improvements in
its programmability, especially high level languages (like C or C++), %
\glspl{GPU} have attracted developers to exploit the computational power of the
hardware for general purpose computing.

%* Problem statement: What problem are you trying to solve? What is the scope of your work (a generalized approach, or for a specific situation)? Be careful not to use too much jargon. In some cases it is appropriate to put the problem statement before the motivation, but usually this only works if most readers already understand why the problem is important.
Several \gls{GPU} programming interfaces and \glspl{API} represent a graphics
centric programming model to developers that is exported by a device driver and
tuned for real time graphics and games. Porting non-graphics applications to
graphics hardware means developing against the graphics programming model. Not
only the difficulties of the unusual graphics centric programming model but also
limitations of the hardware makes development of non-graphics applications a
tedious task.

%* Approach: How did you go about solving or making progress on the problem? Did
%you use simulation, analytic models, prototype construction, or analysis of
%field data for an actual product? What was the extent of your work (did you look
%at one application program or a hundred programs in twenty different programming
%languages?) What important variables did you control, ignore, or measure?
Therefore NVIDIA Corporation developed the \gls{CUDA} that is a fundamentally new
computing architecture that simplifies software development by using the
standard C language. Using \gls{CUDA} this thesis will show on the basis of an
massively parallel application in which extent \glspl{GPU} are suitable for
general purpose computation. Special attention is paid to performance,
computational concepts, efficient data structures and program optimization.

%* Results: What's the answer? Specifically, most good computer architecture papers conclude that something is so many percent faster, cheaper, smaller, or otherwise better than something else. Put the result there, in numbers. Avoid vague, hand-waving results such as "very", "small", or "significant." If you must be vague, you are only given license to do so when you can talk about orders-of-magnitude improvement. There is a tension here in that you should not provide numbers that can be easily misinterpreted, but on the other hand you don't have room for all the caveats.
The result of this work is the demonstration of feasibility of \gls{GPGPU}. It
will show that \glspl{GPU} are capable of accelerating specific applications by
an order of magnitude.

%* Conclusions: What are the implications of your answer? Is it going to change the world (unlikely), be a significant "win", be a nice hack, or simply serve as a road sign indicating that this path is a waste of time (all of the previous results are useful). Are your results general, potentially generalizable, or specific to a particular case?
This work will represent a general guideline for suggestions and hints as 
well as drawbacks and obstacles when porting applications to \glspl{GPU}.


\vfill