%!TEX TS-program = pdflatex
%!TEX encoding = UTF-8 Unicode
%!TEX TS-options = -halt-on-error
%!TEX TS-options = -shell-escape
\chapter{Proposal}
\label{ch:proposal}

\section*{Scope of the Work} % (fold)
\label{sec:scope_of_the_work}

The first part of the thesis will provide a brief and gentle overview of general
purpose computing on graphics hardware (GPGPU). It will state why \gls{GPGPU}  is
important and address why one should work on the topic and why somebody else
should be interested. After the first part the reader will know what the thesis
is all about, why, what the major issues are and what the proposed solution is.

The second part, the related work, will justify that the problem exists by
example and argument. It will identify and evaluate past approaches to the
problem. The relevance and importance of the work will be presented as a
motivation for interest. The important issues will be identified and background
to the solution will be provided.

In the next part the reader will have enough background so the detailed problem
analysis and solution proposal can be stated. Experiments will resolve
outstanding issues in the solution.

The next part is taking the outstanding issues the experimental results and
analyzing them. By the end of this part the reader should know how the proposed
solution worked out.

The last part will be used to discuss and explain the results, recap the thesis
and conclustons will be drawn based on the analysis. Significant issues
identified, or still outstanding will be described as future work.



