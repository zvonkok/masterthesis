%!TEX TS-program = pdflatex
%!TEX encoding = UTF-8 Unicode

\chapter{Introduction}\label{ch:introduction}

%Technology trends, streaming computation, future
%Industrietendenzen


%\begin{aenumerate}
%	\item general intro into the project
%	\item description of goal
%	\item motivation
%	\item overview of report
%\end{aenumerate}

%A reader of the introduction should be able to answer the following questions, although not in any depth.

%    * What is the thesis about?
%    * Why is it relevant or important?
%    * What are the issues or problems?
%    * What is the proposed solution or approach?
%    * What can one expect in the rest of the thesis?

%State what the thesis is about early. Don't keep the reader guessing until the end of the introduction, or worse, the end of the thesis (don't laugh, I have read draft theses that left me wondering after reading the entire document). You should provide a brief and gentle overview of the thesis topic (or problem) to give the reader enough context  to understand the rest of the introduction. Don't overwhelm the reader with detail at the start. You will provide the details later elsewhere in the thesis. Target the level of writing at one of your peers, but not necessarily somebody working in the same area.

%State why the topic is important. Address the "so what?" criteria. Why are you working on the topic? Why should somebody else be interested? Your motivation should be obvious after the introduction, but not necessarily provably so at this point.

%State what the major issues are in solving your problem. Coherently overview the issues in enough detail to be able to understand they exist, but don't go into details yet or attempt to prove they exist. The overview should be in just enough depth to understand why you might propose the your particular solution or approach you are taking.

%Describe your proposed solution or position your taking. Again, you should not go into minute details, nor should you attempt to prove your solution at this point; the remainder of the thesis will describe and substantiate your solution in detail, that what a thesis is :-)

%At this point the reader will know what your working on, why, what are the major issues, and what your proposed solution is, but usually only if he takes your word for it. You should outline what the reader should expect in the rest of the thesis. This is not just the table of contents in sentence form, it is an overview of the remainder of the thesis so the reader knows what to expect.

% section introduction (end)
%This section should briefly overview the provided topic.

% What the subject is; general purpose computing on gpus -----------------------

In recent years \glspl{GPU} have moved from fixed pipeline processors to fully
programmable processors. This evolvement has attracted many developers to do
\gls{GPGPU}. The \glspl{GPU} have devoted there silicon (transistors) for
computing engines rather than for control engines like caches, branch
prediction, coherency protocols compared to traditional \glspl{CPU}. This
incredible computing power made algorithms with an high arithmetic density run
by an order of magnitude faster than on \glspl{CPU}. Speedups of 
100$\times$--400$\times$ faster than the \gls{CPU} were stunning but only a few
people understand why such speedups are possible and why only a couple of
algorithm can attain such speedups.

% why you are investigating it -------------------------------------------------

Many developers in these days are faced with multicore processors and have to
implement or extend existing algorithms to take full advantage of the processing
power of such cores. The \gls{CPU} manufacturers are facing fundamental problems
when increasing performance only by frequency. In former times higher frequency
meant higher performance but a paradigm shift took place now the new way is
going multicore which means higher performance. Moore's law says that for every
2 years the number of transistors on a chip will double. The transistors will be
dedicated to more cores rather than for bigger chips as in the past. What does
it mean to developers? They have to think in parallel, not only for two or four
cores but rather for 16 or 32 cores. They have to assure that there code is
scaling over many cores over many generations of \gls{CPU} chips. There are
several parallel programming languages and middleware to help developers to
program in parallel but a quasi standard has not been established. 

% which aspects you will consider, and why -------------------------------------

There are several programming platforms that are favoured in certain fields 
respectively system levels. On thread or loop-level there is OpenMP a programming
interface for parallelisation on shared-memory systems. Going a level higher to
distributed systems that exchange messages during parallel execution there is the
\gls{MPI} which is a quasi standard for message passing. There are further 
libraries but none of them is really suited for heterogenous computing systems.

Heterogeneous computing systems are different than common computing systems. For
developers it is crucial to understand the underlying architecture and
programming models. They have to find the right tool (\gls{API}, middleware,
\ldots) to exploit parallelism. Pitfalls and drawbacks often arise when studying
the new system and developers have to adapt and deal with it. Furthermore its
not only the use of the right tools its more the parallel thinking which is new
to most of them. Parallel thinking and spotting concurrency in algorithms is the
key for acceleration. Those aspects have to be considered when trying to do
\gls{GPGPU}.

% which aspects you will not consider, and why ---------------------------------

As stated before there are several programming models respectively \glspl{API} 
available for programming the \gls{GPU}. Most of them are based on a graphics 
\gls{API} and are exporting a C++ interface to the developer abstracting
the underlying hardware. The biggest problem with such \glspl{API} is that 
there are a lot of them and every has its pros and cons. Thats why the focus
will be on \glspl{API} directly supplied by the manufacturers of the \glspl{GPU}.

% what you hope to find out ----------------------------------------------------
At the time of writing this thesis there were several myths about \gls{GPGPU}.
Developers are baffled when they hear they have to implement the algorithm with
a graphics \gls{API} or it is not possible to do scatter operations to the memory.
There exist further myths that discourages developers from computing on
\glspl{GPU} that should be demystified. 

New technologies have a higher learning curve than existing and its always the 
question if its feasible and if its even possible to do general purpose computing
on \glspl{GPU}. 

% what your starting point(s) will be ------------------------------------------

The starting point will be a review of what has been done followed by a
feasibility study which will demystify the myths and cover major obstacles and
drawbacks when doing computations on a \gls{GPU}. Further experiments will show
and reflect the computational capabilities especially in the view of general 
purpose programs. 

% what assumptions you are making ----------------------------------------------

The \gls{GPU} compared to the \gls{CPU} is very powerful and inexpensive. It has
a much higher memory bandwidth and computational throughput. Not only that it is
fast it is accelerating quickly. Recent improvements made the \gls{GPU} as well
flexible and programmable. But there are limitations and difficulties. It is not
only the new programming model or architecture the massively parallel nature of 
a \gls{GPU} discomforts many developers. 

% how you will present the subject ---------------------------------------------

The thesis will therefore start with a background chapter to show the relevance
of the topic. After establishing the knowledge the reader will get a first
introduction into the architecture and programming model of \glspl{GPU}.
Equipped with the knowledge a first algorithm will be ported to the \gls{GPU}
for experiments and a first hands on.

This preparation will be helpful when getting to the main part where the actual
algorithm is presented. An extensive analysis followed by the design and implementation
of the algorithm will complete the theoretical part. 

The following chapter will then go deeper into hardware specific stuff where the
algorithm will be highly optimised for the \gls{GPU}. The optimisation will be 
verified in a subsequent chapter where the performance and scalability of the
algorithm will be presented. 

Last but not least a summary and future work will top the thesis off. 

% the aims and objectives of the work presented in the report ------------------
\section{Aims \textit{\&} Objectives} % (fold)
\label{sec:aims_objectives}

This thesis will cover all the topics to understand the architecture,
programming model, drawback and pitfalls when doing \gls{GPGPU}. The \gls{GPU}
is a highly parallel processor with thousands of threads and a peak performance
of several hundred \glspl{GFLOP}.

By the means of an application which will be ported to the \gls{GPU} the general
workflow will be shown and various procedure models examined. It will be
presented that often traditional software engineering principles do not apply to
high performance, parallel computing. For this work a \slcsmallcaps{NVIDIA} \gls{GPU}
will be used together with \gls{CUDA} which is a extension to C for parallel
programming of \glspl{GPU}.

The remainder of the thesis will give some in depth background to the topic and
expose the programming models and the architecture of \glspl{GPU}. Furthermore the
development of a parallel implementation of an algorithm will be examined step
by step. In this context software analysis and design principles will be shown
that fit to parallel programming. A feasibility study will cover major obstacles
and show how to avoid them.

Finally an application for segmenting an image will be implemented and presented
in all aspects to the reader.
% section aims_\&_objectives (end)

% the initial time plan for the project work -----------------------------------
\section{Initial Time plan} 
\label{sub:time_plan} 
%Strongly related to the key activities identified above.
For this thesis the following milestones were defined and listed below. Each
milestone will have a short description which tasks have to be accomplished to
complete a mile stone. With every milestone a chapter respectively a section of
the master thesis will be written which corresponds to the current carried out
milestone. So the development of the milestones goes hand in hand with editing
of the thesis.

\paragraph{M1: Related Work Identified (2/14/2009)} % (fold)
\label{par:m1_related_work_identified}
The next step to take is to identify related work, conducting the literature 
survey. This survey should show the demand of \gls{GPGPU} and used algorithms. 
Furthermore it will help to understand the architecture and programming models. 
After gaining enough knowledge about the field its time to get the hands on the 
\gls{GPU}.
% paragraph m1_related_work_identified (end)
\paragraph{M2: Algorithmic view understood (2/28/2009)} % (fold)
\label{par:m2_architecture_algorithmic_view_understood}
Equipped with the knowledge from the latter milestone it is time to understand
how the architecture works by examining algorithms already mapped to the \gls{GPU}.
This step is crucial as it shows how algorithms map to and work on the \gls{GPU}
and a good understanding of the environment can help later on for choosing the
right algorithm for porting, analysis, design and implementation.
% paragraph m2_architecture_algorithmic_view_understood (end)
\paragraph{M3: Feasibility Study (3/14/2009)} % (fold)
\label{par:m3_feasibility_study}
The feasibility study will be used for the first steps into \gls{GPGPU} . A low featured
ray tracer will be used for examining what the hurdles and obstacles are when 
developing for the \gls{GPU}. The knowledge will be used for choosing the right 
algorithm. 
% paragraph m3_feasibility_study (end)
\paragraph{M4: Application, Algorithm chosen (3/21/2009)} % (fold)
\label{par:m4_application_algorithm_choosen}
After completion of this milestone the algorithm is chosen to be ported to the 
\gls{GPU}. So the next milestones will concentrate only on this algorithm and will
not consider the general case. 
% paragraph m4_application_algorithm_choosen (end)
\paragraph{M5: Analysis, Solution Proposed (4/4/2009)} % (fold)
\label{par:m5_analysis_solution_proposed}
Here a solution will be proposed and outstanding issues will be identified. 
There are several myths about \gls{GPGPU}  which will be stated here as well, because 
there are affecting the design of the application and cannot be discarded.
% paragraph m5_analysis_solution_proposed (end)
\paragraph{M6: Experiments (4/14/2009)} % (fold)
\label{par:m6_experiments}
To exclude all eventualities a set of experiments will be undertaken. This will 
as well solve all myths.
% paragraph m6_experiments (end)
\paragraph{M7: Design, Application Ported (5/14/2009)} % (fold)
\label{par:m7_design_application_ported}
This milestone will cover the design of the application with all constraints,
obstacles and hurdles that are involved with \gls{GPGPU} . In this task the application
will be implemented and run on the \gls{GPU}. A good implementation will need several
iterations were several aspects will be tuned: coalesced memory access, load
balancing and many more.
% paragraph m7_design_application_ported (end)
\paragraph{M8: Discussion (5/22/2009)} % (fold)
\label{par:m8_discussion}
Another important part is the discussion of the results and methodologies
applied which will be done in this milestone.
% paragraph m8_discussion (end)

\paragraph{M9: Master Thesis Edited (5/30/2009)} % (fold)
\label{par:master_thesis_edited}
Here the thesis will be recapped and briefly outlined. Conclusions will be drawn
and future work presented. A critical summary and conclusion will be provided as
well. % paragraph master_thesis_edited (end)


 



% section introduction_ir (end)
