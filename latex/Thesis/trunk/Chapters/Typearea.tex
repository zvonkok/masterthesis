\chapter{Typearea}

In recent years graphic processing units (GPUs) have moved from fixed pipeline
graphics processors to a fully programmable processor. This evolvment has
attracted many developers to do general purpose computing on GPUs. \glspl{GPU}have
devoted there sillicon (transistors) for computing engines rather than for
control engines like caches, branch prediction, coherency protocols and more.
This incredible computing power made algorithms with an high arithemtic density
run by an order of magnitude faster than on central processing units (CPUs).
Speedups of 100x faster than the CPU were stunning but only a few people
understand why such speedups are possible and why only a couple of algorithm can
attain such speedups.

This thesis will cover all the topics to understand the architecture,
programming model, software eco system, drawback and pittfalls when doing
general purpose computing on GPUs. The \gls{GPU} is a highly parallel processor with
thousands of threads and a peak performance of ~600 GFlops (G92
core\footnote{http://www.nvidia.com/page/geforce\_8800.html}).

Many developers in these days are faced with multicore processors and have to
implement or extend existing algorithms to take full advantage of the
processing power of such cores. CPU manufacturer are facing fundamental problems
when increasing performance only by frequency. In former times higher frequency
meant higher performance but a paradigm shift took place now the new stigma is
more cores means higher performance. Moore's law says that for every 2 years the
amount of cores on a chip will double. What does it mean to developers? They
have to think in parallel, not only for two or four cores but rather for 16 or
32 cores. They have to assure that there code is scaling over many cores over
many generations of CPU chips. There are several parallel programming languages
and middleware to help developers to programm in parallel but a quasi standard
has not been established.

By the means of an application which will be ported to the \gls{GPU} the general
workflow will be shown and various procedure models examined. It will be
presented that often traditional software engineering principles do not
apply to high performance, parallel computing. For this work a Nvidia \gls{GPU} will
be used togehter with \gls{CUDA}(Common Unified Device Architecture) that is a
extension to C for parallel programming of GPUs. 

The remainder of the thesis will give some in depth background to the topic and
expose with programming models and the architecture of GPUs. Furthermore the
development of a parallel implementation of an algorithm will be examined step
by step. In this context software analysis and design principles will be shown
that fit to parallel programming. A feasibility study will cover major obstacles
and show how to avoid them.

Finnaly an application for visualization of complex and chaotic dynamical
systems will be implemented and presented in all aspects to the reader.

In recent years graphic processing units (GPUs) have moved from fixed pipeline
graphics processors to a fully programmable processor. This evolvment has
attracted many developers to do general purpose computing on GPUs. \glspl{GPU}have
devoted there sillicon (transistors) for computing engines rather than for
control engines like caches, branch prediction, coherency protocols and more.
This incredible computing power made algorithms with an high arithemtic density
run by an order of magnitude faster than on central processing units (CPUs).
Speedups of 100x faster than the CPU were stunning but only a few people
understand why such speedups are possible and why only a couple of algorithm can
attain such speedups.

This thesis will cover all the topics to understand the architecture,
programming model, software eco system, drawback and pittfalls when doing
general purpose computing on GPUs. The \gls{GPU} is a highly parallel processor with
thousands of threads and a peak performance of ~600 GFlops (G92
core\footnote{http://www.nvidia.com/page/geforce\_8800.html}).

Many developers in these days are faced with multicore processors and have to
implement or extend existing algorithms to take full advantage of the
processing power of such cores. CPU manufacturer are facing fundamental problems
when increasing performance only by frequency. In former times higher frequency
meant higher performance but a paradigm shift took place now the new stigma is
more cores means higher performance. Moore's law says that for every 2 years the
amount of cores on a chip will double. What does it mean to developers? They
have to think in parallel, not only for two or four cores but rather for 16 or
32 cores. They have to assure that there code is scaling over many cores over
many generations of CPU chips. There are several parallel programming languages
and middleware to help developers to programm in parallel but a quasi standard
has not been established.

By the means of an application which will be ported to the \gls{GPU} the general
workflow will be shown and various procedure models examined. It will be
presented that often traditional software engineering principles do not
apply to high performance, parallel computing. For this work a Nvidia \gls{GPU} will
be used togehter with \gls{CUDA}(Common Unified Device Architecture) that is a
extension to C for parallel programming of GPUs. 

The remainder of the thesis will give some in depth background to the topic and
expose with programming models and the architecture of GPUs. Furthermore the
development of a parallel implementation of an algorithm will be examined step
by step. In this context software analysis and design principles will be shown
that fit to parallel programming. A feasibility study will cover major obstacles
and show how to avoid them.

Finnaly an application for visualization of complex and chaotic dynamical
systems will be implemented and presented in all aspects to the reader.

In recent years graphic processing units (GPUs) have moved from fixed pipeline
graphics processors to a fully programmable processor. This evolvment has
attracted many developers to do general purpose computing on GPUs. \glspl{GPU}have
devoted there sillicon (transistors) for computing engines rather than for
control engines like caches, branch prediction, coherency protocols and more.
This incredible computing power made algorithms with an high arithemtic density
run by an order of magnitude faster than on central processing units (CPUs).
Speedups of 100x faster than the CPU were stunning but only a few people
understand why such speedups are possible and why only a couple of algorithm can
attain such speedups.

This thesis will cover all the topics to understand the architecture,
programming model, software eco system, drawback and pittfalls when doing
general purpose computing on GPUs. The \gls{GPU} is a highly parallel processor with
thousands of threads and a peak performance of ~600 GFlops (G92
core\footnote{http://www.nvidia.com/page/geforce\_8800.html}).

Many developers in these days are faced with multicore processors and have to
implement or extend existing algorithms to take full advantage of the
processing power of such cores. CPU manufacturer are facing fundamental problems
when increasing performance only by frequency. In former times higher frequency
meant higher performance but a paradigm shift took place now the new stigma is
more cores means higher performance. Moore's law says that for every 2 years the
amount of cores on a chip will double. What does it mean to developers? They
have to think in parallel, not only for two or four cores but rather for 16 or
32 cores. They have to assure that there code is scaling over many cores over
many generations of CPU chips. There are several parallel programming languages
and middleware to help developers to programm in parallel but a quasi standard
has not been established.

By the means of an application which will be ported to the \gls{GPU} the general
workflow will be shown and various procedure models examined. It will be
presented that often traditional software engineering principles do not
apply to high performance, parallel computing. For this work a Nvidia \gls{GPU} will
be used togehter with \gls{CUDA}(Common Unified Device Architecture) that is a
extension to C for parallel programming of GPUs. 

The remainder of the thesis will give some in depth background to the topic and
expose with programming models and the architecture of GPUs. Furthermore the
development of a parallel implementation of an algorithm will be examined step
by step. In this context software analysis and design principles will be shown
that fit to parallel programming. A feasibility study will cover major obstacles
and show how to avoid them.

Finnaly an application for visualization of complex and chaotic dynamical
systems will be implemented and presented in all aspects to the reader.

In recent years graphic processing units (GPUs) have moved from fixed pipeline
graphics processors to a fully programmable processor. This evolvment has
attracted many developers to do general purpose computing on GPUs. \glspl{GPU}have
devoted there sillicon (transistors) for computing engines rather than for
control engines like caches, branch prediction, coherency protocols and more.
This incredible computing power made algorithms with an high arithemtic density
run by an order of magnitude faster than on central processing units (CPUs).
Speedups of 100x faster than the CPU were stunning but only a few people
understand why such speedups are possible and why only a couple of algorithm can
attain such speedups.

This thesis will cover all the topics to understand the architecture,
programming model, software eco system, drawback and pittfalls when doing
general purpose computing on GPUs. The \gls{GPU} is a highly parallel processor with
thousands of threads and a peak performance of ~600 GFlops (G92
core\footnote{http://www.nvidia.com/page/geforce\_8800.html}).

Many developers in these days are faced with multicore processors and have to
implement or extend existing algorithms to take full advantage of the
processing power of such cores. CPU manufacturer are facing fundamental problems
when increasing performance only by frequency. In former times higher frequency
meant higher performance but a paradigm shift took place now the new stigma is
more cores means higher performance. Moore's law says that for every 2 years the
amount of cores on a chip will double. What does it mean to developers? They
have to think in parallel, not only for two or four cores but rather for 16 or
32 cores. They have to assure that there code is scaling over many cores over
many generations of CPU chips. There are several parallel programming languages
and middleware to help developers to programm in parallel but a quasi standard
has not been established.

By the means of an application which will be ported to the \gls{GPU} the general
workflow will be shown and various procedure models examined. It will be
presented that often traditional software engineering principles do not
apply to high performance, parallel computing. For this work a Nvidia \gls{GPU} will
be used togehter with \gls{CUDA}(Common Unified Device Architecture) that is a
extension to C for parallel programming of GPUs. 

The remainder of the thesis will give some in depth background to the topic and
expose with programming models and the architecture of GPUs. Furthermore the
development of a parallel implementation of an algorithm will be examined step
by step. In this context software analysis and design principles will be shown
that fit to parallel programming. A feasibility study will cover major obstacles
and show how to avoid them.

Finnaly an application for visualization of complex and chaotic dynamical
systems will be implemented and presented in all aspects to the reader.
In recent years graphic processing units (GPUs) have moved from fixed pipeline
graphics processors to a fully programmable processor. This evolvment has
attracted many developers to do general purpose computing on GPUs. \glspl{GPU}have
devoted there sillicon (transistors) for computing engines rather than for
control engines like caches, branch prediction, coherency protocols and more.
This incredible computing power made algorithms with an high arithemtic density
run by an order of magnitude faster than on central processing units (CPUs).
Speedups of 100x faster than the CPU were stunning but only a few people
understand why such speedups are possible and why only a couple of algorithm can
attain such speedups.
In recent years graphic processing units (GPUs) have moved from fixed pipeline
graphics processors to a fully programmable processor. This evolvment has
attracted many developers to do general purpose computing on GPUs. \glspl{GPU}have
devoted there sillicon (transistors) for computing engines rather than for
control engines like caches, branch prediction, coherency protocols and more.
This incredible computing power made algorithms with an high arithemtic density
run by an order of magnitude faster than on central processing units (CPUs).
Speedups of 100x faster than the CPU were stunning but only a few people
understand why such speedups are possible and why only a couple of algorithm can
attain such speedups.

This thesis will cover all the topics to understand the architecture,
programming model, software eco system, drawback and pittfalls when doing
general purpose computing on GPUs. The \gls{GPU} is a highly parallel processor with
thousands of threads and a peak performance of ~600 GFlops (G92
core\footnote{http://www.nvidia.com/page/geforce\_8800.html}).

Many developers in these days are faced with multicore processors and have to
implement or extend existing algorithms to take full advantage of the
processing power of such cores. CPU manufacturer are facing fundamental problems
when increasing performance only by frequency. In former times higher frequency
meant higher performance but a paradigm shift took place now the new stigma is
more cores means higher performance. Moore's law says that for every 2 years the
amount of cores on a chip will double. What does it mean to developers? They
have to think in parallel, not only for two or four cores but rather for 16 or
32 cores. They have to assure that there code is scaling over many cores over
many generations of CPU chips. There are several parallel programming languages
and middleware to help developers to programm in parallel but a quasi standard
has not been established.

By the means of an application which will be ported to the \gls{GPU} the general
workflow will be shown and various procedure models examined. It will be
presented that often traditional software engineering principles do not
apply to high performance, parallel computing. For this work a Nvidia \gls{GPU} will
be used togehter with \gls{CUDA}(Common Unified Device Architecture) that is a
extension to C for parallel programming of GPUs. 

The remainder of the thesis will give some in depth background to the topic and
expose with programming models and the architecture of GPUs. Furthermore the
development of a parallel implementation of an algorithm will be examined step
by step. In this context software analysis and design principles will be shown
that fit to parallel programming. A feasibility study will cover major obstacles
and show how to avoid them.



\printinunitsof{mm}
\settowidth{\abcd}{abcdefghijklmnopqrstuvwxyz} 	
\begin{table}[ht]
   \myfloatalign
  \begin{tabularx}{\textwidth}{XXXc} \toprule
    \tableheadline{Parameter} & 
	\tableheadline{In pt} & 
	\tableheadline{In mm} & 
	\tableheadline{Comment}\\ \midrule
	abcd... & \the\abcd & \prntlen{\abcd} & \\ % prints the value of the length 
	paperwidth & \the\paperwidth & \prntlen{\paperwidth} & \\
	paperheight & \the\paperheight & \prntlen{\paperheight} & \\
	textwidth & \the\textwidth & \prntlen{\textwidth} & \\
	textheight & \the\textheight & \prntlen{\textheight} & \\
	evensidemargin &   \the\evensidemargin & \prntlen{\evensidemargin} & \\
	oddsidemargin &  \the\oddsidemargin  & \prntlen{\oddsidemargin} & \\
 	paperheight & \the\paperheight & \prntlen{\paperheight} & \\
	topmargin & \the\topmargin  & \prntlen{\topmargin} & \\
	headheight & \the\headheight & \prntlen{\headheight} & \\
 	headsep & \the\headsep  & \prntlen{\headsep} & \\
 	topskip & \the\topskip  & \prntlen{\topskip} & \\
	footskip & \the\footskip  & \prntlen{\footskip} & \\
	baselineskip & \the\baselineskip  & \prntlen{\baselineskip} & \\	
    \bottomrule
  \end{tabularx}
  \caption[Type area calculation]{Type area calculation}
  \label{tab:comp}
\end{table}
%	%
%	(typearea)             DIV  = areaset
%	(typearea)             BCOR = 14.22636pt
%	(typearea)             \paperwidth      = 597.50793pt
%	(typearea)              \textwidth      = 348.0pt
%	(typearea)              \evensidemargin = 84.58438pt
%	(typearea)              \oddsidemargin  = 20.38356pt
%	(typearea)             \paperheight     = 845.04694pt
%	(typearea)              \textheight     = 616.04997pt
%	(typearea)              \topmargin      = -23.92102pt
%	(typearea)              \headheight     = 15.95pt
%	(typearea)              \headsep        = 20.40001pt
%	(typearea)              \topskip        = 11.0pt
%	(typearea)              \footskip       = 47.60002pt
%	(typearea)              \baselineskip   = 13.6pt
%	(typearea)              on input line 268.
	
%*******************************************************************************	
